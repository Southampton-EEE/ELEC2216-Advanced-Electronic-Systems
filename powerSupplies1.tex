\section{Power Supplies}

\begin{center}
\textit{Power Supplies 1}
\vspace{1mm}
\hrule
\end{center}

\subsection{What \textit{is} power?}

Power is the rate of disapation of energy. You can express power in terms of voltage and current using the chain rule.

\begin{align*}
P = \frac{dE}{dt},\, V = \frac{dP}{dQ},\, I = \frac{dQ}{dt}
\end{align*}

\begin{align*}
P = I \cdot V = \frac{dP}{dQ} \cdot \frac{dQ}{dt} = \frac{dE}{dt}
\end{align*}

Losses are often expressed in terms of power. Ohmic loss is the power lost though resistance. 

\begin{align*}
P = I^2R
\end{align*}

This means that to decrease Ohmic loss you can increase the voltage while decreasing the current. This is easy to do for AC with a transformer, but harder to achieve for DC.

\subsection{Quantifying Time-dependent Voltage}

\begin{itemize}
	\item Instantaneous value
	\item Peak value
	\item Peak to peak value
	\item Average value
	\item Effective value or RMS
\end{itemize}

\subsubsection{Effective Voltage and Power}

$V_{RMS}$ is a voltage that will cause the same loss in a resistance $R$ as a DC voltage $V$.

\begin{align*}
P = IV = \frac{V^2}{R} \\
P_{AV} = \frac{1}{T} \int^_
\end{align*} 